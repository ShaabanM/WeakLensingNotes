\subsection{Notes}
\begin{enumerate}
    \item Motivate The Problem Cosmo
    \item Introduce The Theory Cosmo + Lensing
    \item Explain How Theory Solves The Problem
    \item Current Results/ Experiments
    \item Explain Problems with WeakLensing
    \item Summarise
\end{enumerate}

Discuss probing the large scale universe with CMB, supernvae, clustering
and lensing. 
Note weak lensing makes no assumptions about the nature of dark matter
and no assumptions about relationship between visible matter and mass therefore
it provides a directly measured mass distribution in the universe as a function of 
redshift. Therefore we can get info on DE and DM directly. It is sensitive to intial
conditions so it can even give info on inflation. 


\subsection{Bending of Light}
The fundamental concept on which weak lensing is built is gravity's ability to alter the path of a photon.
In this section we review the theory behind the bending of light necessary to develop the weak lensing formalism.
\subsubsection{Newtonian Lens}

It is a common misconception that the gravitational bending of light is an exclusive property of GR.
However, gravity induced alterations to a photon's path are predicted by newtonian mechanics \cite{lensingbook}. To illustrate this 
consider a mass $M$ located at the origin of the cartesian plane and a corpuscle(newtonian photon) 
propagating along the $x=b$ line. 
Newton's second law predicts that the presence of the point mass will result in a momentum transfer
between the two objects. If the corpuscle starts with 
momentum $(p,0)$ then it will end up with momentum $(p_x,p_y)$.
Therefore, the particle path is deflected by some angle $\Delta \theta$. The deflection angle is 
simply given by 

\begin{equation}
  \sin(\Delta \theta) = \frac{p_y}{\sqrt{p_x^2+p_y^2}}
  \label{deflectionnewton}
\end{equation}

\par For very small deflections we have $p\approx p_x >> p_y$ and $\Delta \theta << 1$. 
Therefore \autoref{deflectionnewton} simplifies to $\Delta \theta
\approx \frac{p_y}{p_x}$. We now consider the infinitesimal deflection along the entire path of the photon with
$d\Delta \theta = \frac{dp_y}{p_x} = \frac{1}{px} dx \frac{dp_y}{dx}$. Therefore, we can find the deflection
angle by 

\begin{equation}
  \begin{split}
  \Delta \theta_N &= -\frac{1}{p_x} \int dx \frac{dp_y}{dx} \\
  &= -\frac{1}{cp_x} \int dx \frac{dp_y}{dt} = -\frac{1}{cp_x} \int F_y dx \\ 
  &= \frac{2GM}{c^2b}
  \end{split}  
  \label{newtonbend}
\end{equation}

We note that the mass of the corpuscle cancels out of the deflection equation. Therefore this equation applies
for massless particles i.e. photons. Therefore \autoref{newtonbend} provides a newtonian description for the 
bending of light \cite{lensingbook}.

\subsubsection{General Relativistic Bending of Light}
The Einstein's field equations in the presence of a charge free point mass is uniquely solved by 
the Schwarzchild Metric 
\cite{}.

\begin{equation}
  \Delta \theta = 2\Delta \theta_N=\frac{4GM}{c^2b}
  \label{grbend}
\end{equation}

\subsection{Lensing Formalism}
\cite{basicLens}
