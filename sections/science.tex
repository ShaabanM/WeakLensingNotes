\subsection{Cosmic Shear}

\subsection{Cosmic Tomography}
For current and future surveys, one goal is to use the redshifts of the background galaxies (often approximated using photometric redshifts) to divide the survey into multiple redshift bins. The low-redshift bins will only be lensed by structures very near to us, while the high-redshift bins will be lensed by structures over a wide range of redshift. This technique, dubbed "cosmic tomography", makes it possible to map out the 3D distribution of mass. Because the third dimension involves not only distance but cosmic time, tomographic weak lensing is sensitive not only to the matter power spectrum today, but also to its evolution over the history of the universe, and the expansion history of the universe during that time. This is a much more valuable cosmological probe, and many proposed experiments to measure the properties of dark energy and dark matter have focused on weak lensing, such as the Dark Energy Survey, Pan-STARRS, and Large Synoptic Survey Telescope.
\cite{lensingbook} \cite{rachel_2018} \cite{hoekstra}
