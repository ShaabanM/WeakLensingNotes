\documentclass{article}
\usepackage[utf8]{inputenc}

\title{Weak Lensing for Precision Cosmology}
\author{Mohamed Shaaban}
\date{March 2019}

\usepackage{natbib}
\usepackage{graphicx}
\usepackage{amsmath}
\usepackage{hyperref}
\begin{document}

\maketitle

\section{Introduction}
The most revolutionary discovery in cosmology since 
Hubble observed that the Universe is expanding is that 
this expansion is accelerating. A revelation that was 
awarded the 2011 Nobel Prize for its profound 
implications. \cite{nobel}. An accelerating
universe implies that either our understanding of gravity is flawed 
or that a mysterious pressure known as Dark Energy is driving the 
expansion \cite{peebles}.
This Dark Energy accounts for most (over 68\%) of the energy density in the observable universe, 
however its origin and physics are presently unknown \cite{planck}. 
As a result, the nature of Dark Energy is considered one of the 
greatest mysteries of modern science \cite{pathfinder}.  
\par
One of the most powerful techniques to probe Dark Energy and modified theories of gravity is weak lensing. Insert discription of weak lensing here \cite{hoekstra}.

\section{Background Theory}
% \subsection{Notes}
% \begin{enumerate}
%     \item Motivate The Problem Cosmo
%     \item Introduce The Theory Cosmo + Lensing
%     \item Explain How Theory Solves The Problem
%     \item Current Results/ Experiments
%     \item Explain Problems with WeakLensing
%     \item Summarise
% \end{enumerate}

% Discuss probing the large scale universe with CMB, supernvae, clustering
% and lensing. 
% Note weak lensing makes no assumptions about the nature of dark matter
% and no assumptions about relationship between visible matter and mass therefore
% it provides a directly measured mass distribution in the universe as a function of 
% redshift. Therefore we can get info on DE and DM directly. It is sensitive to intial
% conditions so it can even give info on inflation. 
\subsection{Standard Model of Cosmology}
The fundamental assumption in cosmology, known as the cosmological principle, is that we live in a homogenous and isotropic universe \cite{general_2013}. 

\subsubsection{Matter Power Spectrum}
\cite{extragalactic}

\subsection{Bending of Light}
The fundamental concept on which weak lensing is built is gravity's ability to alter the path of a photon.
In this section we review the theory behind the bending of light necessary to develop the weak lensing formalism.
\subsubsection{Newtonian Lens}
\label{subsec:newtonlens}

It is a common misconception that the gravitational bending of light is an exclusive property of GR.
However, gravity induced alterations to a photon's path are predicted by newtonian mechanics \cite{lensingbook}. To illustrate this 
consider a mass $M$ located at the origin of the cartesian plane and a corpuscle(newtonian photon) 
propagating along the $x=b$ line (in this context $b$ is known as the impact parameter). 
Newton's second law predicts that the presence of the point mass will result in a momentum transfer
between the two objects. If the corpuscle starts with 
momentum $(p,0)$ then it will end up with momentum $(p_x,p_y)$.
Therefore, the particle path is deflected by some angle $\Delta \theta$. The deflection angle is 
simply given by 

\begin{equation}
  \sin(\Delta \theta) = \frac{p_y}{\sqrt{p_x^2+p_y^2}}
  \label{deflectionnewton}
\end{equation}


\par For very small deflections we have $p\approx p_x >> p_y$ and $\Delta \theta << 1$. 
Therefore \autoref{deflectionnewton} simplifies to $\Delta \theta
\approx \frac{p_y}{p_x}$. We now consider the infinitesimal deflection along the entire path of the photon with
$d\Delta \theta = \frac{dp_y}{p_x} = \frac{1}{px} dx \frac{dp_y}{dx}$. Therefore, we can find the deflection
angle by 

\begin{equation}
  \begin{split}
  \Delta \theta_N &= -\frac{1}{p_x} \int dx \frac{dp_y}{dx} \\
  &= -\frac{1}{cp_x} \int dx \frac{dp_y}{dt} = -\frac{1}{cp_x} \int F_y dx \\ 
  &= \frac{2GM}{c^2b}
  \end{split}  
  \label{newtonbend}
\end{equation}

We note that the mass of the corpuscle cancels out of the deflection equation. Therefore this equation applies
for massless particles i.e. photons. Therefore \autoref{newtonbend} provides a newtonian description for the 
bending of light \cite{lensingbook}.

\subsubsection{General Relativistic Bending of Light}
In this subsubsection I give a quick sketch of the bending of light in the context of general relativity,
for a more detailed calculation please consult \cite{GR1}.

\par The Einstein's field equations in the presence of a charge free static point mass is uniquely solved by 
the Schwarzchild metric \cite{GR1}. The Schwarzchild metric is

\begin{equation}
  ds^2 = \left ( 1-\frac{r_s}{r} \right )  dt^2 - \left( 1-\frac{r_s}{r}\right) ^{-1} dr^2 -r^2 d\Omega^2
  \label{schwarz}
\end{equation}

Where $r_s$ is the Shcwarzchild radius of the system given by $r_s=2 \mu = 2GM/c^2$. We can analyze the path of the photon from \autoref{subsec:newtonlens} by studying the geodesic equations of the metric and finding the conserved
quantities of the system. We can then combine the conservation equations with the tangent vector norm condition for a 
null path to get the shape equation of the system as 

\begin{equation}
  \frac{d\phi}{dr} = \frac{1}{r^2} \left(\frac{1}{b^2}- \frac{1}{r^2} \left(1-\frac{2\mu}{r}\right) \right)^{-1/2}
  \label{shapeeqbend}
\end{equation}

where $(r,\phi)$ are the photons position in 2D polar coordinates and $b$ is the impact parameter. Rewriting this equation
under the transformation of $r = 1/u$ and working pertrubatively around $u(\mu =0) = \frac{1}{b}\sin \phi
$ we get 

\begin{equation}
  u(\phi) \approx \frac{1}{b}\sin \phi + \frac{3\mu}{2b^2} \left(1+\frac{1}{3}\cos 2 \phi \right)
  \label{eq:pertshape}
\end{equation}

in the limit were $\phi << 1$ and $u \rightarrow 0$ \autoref{eq:pertshape} simplifies to $\phi = \Delta \theta_N =\frac{2GM}{c^2b} $. Geometrically the deflection is given by $\Delta \theta = 2\phi$ and therefore the deflection angle is

\begin{equation}
  \Delta \theta = 2\Delta \theta_N=\frac{4GM}{c^2b}
  \label{grbend}
\end{equation}

We conclude that general relativity predicts a factor of 2 greater deflection form a point mass than is predicted by newtonian mechanics. This relationship greatly simplifies the formalism developed for weak lensing. 

\subsection{Weak Lensing Formalism}
\cite{basicLens}
% Cosmology standard model and lensing formalism

% \section{Cosmological Observables}
% Discussion about cosmology in general and observable in the framework \cite{general_2013}.

% \section{Weak Gravitational Lensing}
% Discussion about lensing as a whole then specifics \cite{rachel_2018} \cite{massey_2013} \cite{Subaru_2019} 

\section{Measuring Shear}

\section{Cosmic Shear}
\cite{lensingbook} \cite{rachel_2018} \cite{hoekstra}

\section{Problems With Weak Lensing}
\cite{massey_2013}

\section{Weak Lensing Results}
\cite{Subaru_2019}

\bibliographystyle{plain}
\bibliography{refs}
\end{document}
