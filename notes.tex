\section{Notes}
In this section I present some general notes on weak lensing. 

\subsection{Talk Structure:}
\begin{enumerate}
    \item Motivate The Problem Cosmo
    \item Introduce The Theory Cosmo + Lensing
    \item Explain How Theory Solves The Problem
    \item Current Results/ Experiments
    \item Explain Problems with WeakLensing
    \item Summarise
\end{enumerate}

\subsubsection{Motivate The Problem}
Discuss probing the large scale universe with CMB, supernvae, clustering
and lensing. 
Note weak lensing makes no assumptions about the nature of dark matter
and no assumptions about relationship between visible matter and mass therefore
it provides a directly measured mass distribution in the universe as a function of 
redshift. Therefore we can get info on DE and DM directly. It is sensitive to intial
conditions so it can even give info on inflation. 

\subsection{The Basics of Lensing Arxiv:0304438}



\subsection{Random Lecture}
\begin{equation}
    \vec{\beta} = \vec{\theta} - \vec{\alpha}(\vec{\theta})
\end{equation}

\begin{equation}
  \vec{\alpha}(\vec{\theta}) = \frac{1}{\pi}  \int d^2\theta'  \kappa(\vec{\theta'}) \frac{\vec{\theta}-\vec{\theta'}}{|\vec{\theta}-\vec{\theta'}|^2}
\end{equation}

\subsubsection{Bending of Light}

\subsubsection{Newtonian Lens}

Contrary to the common belief that bending of light is unique to GR, classical Newtonian mechanics actually 
allows for the bending of light. To illustrate this 
consider a mass $M$ located at the origin of the cartesian plane and a particle propagating along the $x=b$ line.
We note that there will occur some momentum transfer due to gravity. The particle starts off with 
momentum $(p,0)$ and ends up with momentum $(p_x,p_y)$.
Therefore, the particle path is deflected by some angle $\Delta \theta$. The deflection angle is 
simply given by $\sin(\Delta \theta) = \frac{p_y}{\sqrt{p_x^2+p_y^2}}$.
\par For very small deflections we have $p\approx p_x >> p_y$ and $\Delta \theta << 1$. Therefore $\Delta \theta
\approx \frac{p_y}{p_x}$. We can now consider the differntial effect on momentum in the y direction along the 
entire path. we have $\frac{dp_y}{p_x} = \frac{1}{px} dx \frac{dp_y}{dx}$. Therefore we can find the deflection
angle by 

\begin{equation}
  \begin{split}
  \Delta \theta &= -\frac{1}{p_x} \int dx \frac{dp_y}{dx} \\
  &= -\frac{1}{cp_x} \int dx \frac{dp_y}{dt} \\ 
  &= \frac{2GM}{c^2b}
  \end{split}  
  \label{newtonbend}
\end{equation}

We note that the mass of the particle cancels out of the deflection equation. Therefore this equation applies
for massless particles i.e. light. Therefore \autoref{newtonbend} provides a newtonian description for the 
bending of light.

\subsubsection{General Relativistic Bending of Light}
  
  